% presentation
\documentclass[compress]{beamer}

% handout
%\documentclass[handout]{beamer}

\mode<beamer>{
  \usetheme{Frankfurt}
  \mode<presentation>
}

\mode<handout>{
  \usepackage{pgfpages}
  \pgfpagesuselayout{2 on 1}[a4paper]
}
\usepackage{fancyvrb}
\usepackage{tikz}
\usetikzlibrary{mindmap,trees,3d,calc}


% other themes
%\usepackage{beamerthemeWarsaw}
%\usepackage{beamerthemeBoadilla}
%\usepackage[compress]{beamerthemeSingapore}


% layout settings
\setbeamertemplate{blocks}[rounded][shadow=false]
\setbeamertemplate{navigation symbols}{}
\setbeamersize{text margin left = 4mm}
\setbeamersize{text margin right = 4mm}
\setbeamercovered{transparent}

% other packages
%\usetheme[Boadilla]{}
\usepackage{calc}
\usepackage[ruled]{algorithm2e}
\usepackage{stmaryrd}
%

%\usepackage{tabularx,booktabs}
\usepackage{booktabs}
%\useinnertheme{rounded}
\usecolortheme{dolphin}
\usecolortheme{rose}
%\setbeamercolor{title}{fg=black,bg=white!80!black}
\setbeamercolor{title}{fg=black,bg=white!80!blue}
\usefonttheme[onlysmall]{structurebold}


\newcommand{\crystal}[5]{%
  \draw[->, >=latex, #5] (#1) --+ (#2);%
  \draw[->, >=latex, #5] (#1) --+ (#3);%
  \draw[->, >=latex, #5] (#1) --+ (#4);%
}

\newcommand{\fillcrystal}[5]{%
  \filldraw[fill=red!25,draw=red!35,opacity=0.5]
%  (0,0) + (1,0) -- (0,0) --++ (0,1) --+ (1,0);
  (#1) +(#2) -- (#1) --++ (#3) -- +(#2) ;%
  \filldraw[fill=red!45,draw=red!35,opacity=0.5]
  (#1) +(#3) -- (#1) --++ (#4) -- +(#3) ;%
  \filldraw[fill=red!65,draw=red!35,opacity=0.5]
  (#1) +(#4) -- (#1) --++ (#2) -- +(#4) ;%
  \draw[->, >=latex, #5] (#1) --+ (#2);%
  \draw[->, >=latex, #5] (#1) --+ (#3);%
  \draw[->, >=latex, #5] (#1) --+ (#4);%
}

\newcommand{\filldirection}[5]{%
  \filldraw[fill=red!25,draw=red!35,opacity=0.5]
%  (0,0) + (1,0) -- (0,0) --++ (0,1) --+ (1,0);
  (#1) +(#2) -- (#1) --++ (#3) -- +(#2) ;%
  \filldraw[fill=red!45,draw=red!35,opacity=0.5]
  (#1) +(#3) -- (#1) --++ (#4) -- +(#3) ;%
  \filldraw[fill=red!65,draw=red!35,opacity=0.5]
  (#1) +(#4) -- (#1) --++ (#2) -- +(#4) ;%
  \draw[->, >=latex, #5] (#1) --+ (#2);%
  \draw[->, >=latex, #5] (#1) --+ (#3);%
  \draw[->, >=latex,color=red, very thick] (#1) --+ (#4);%
}



\usepackage{colortbl}

\usepackage{listings}


\definecolor{lightgray}{gray}{0.9}
\definecolor{lightergray}{gray}{0.92}
\definecolor{mygreen}{rgb}{0.1,0.9,0.3}
\definecolor{commentgreen}{rgb}{0.3,1,0.5}

\definecolor{lightgray}{gray}{0.9} \lstset{backgroundcolor=\color{lightgray}}
\lstdefinelanguage{rock}{morekeywords=
{mean,std,hist,save,load,rotate,min,max,get,tensor,loadTensor,
scale,union,delete,degree,kernel,gethw,getparameter,bandwidth,
symmetry,Miller,Miller2quat,uniformODF,unimodalODF,fibreODF,xvector,yvector,zvector,
loadPoleFigure,PoleFigure,plot,plotpdf,plotipdf,calcODF,calcKernel,calcTensor,calcMDF,calcMisorientation,
vector3d,cross,norm, dot,sum,savefigure,plotDiff,calcerror,
quaternion,idquaternion,dist,Laue,length,angle,add,subGrid,refine,GridLength,getResolution,getRho,
polar,S2Grid,santafee,mix2,modalorientation,fourier,textureindex,entropy,volume,
surface, fibrevolume,uiimport,scatter,cif2symmetry,remove_outlier,
simulatePoleFigure,SimulateEBSD,grainfun,hold,getFundamentalRegion,symvec,
FourierODF,find_outlier,bar,copyproperty,plotspatial,hasholes,principalcomponents,
hullprincipalcomponents,loadEBSD,calcGrains,plotboundaries,plotodf,plotspatial,area,
aspectratio,grain,grainfun,grainsize,get,plotgrains,plotellipse,plotsubfractions,plotboundary,
hasholes,hassubfraction,hullarea,hullcentroid,hullperimeter,centroid,borderlength,
deltaarea,equivalentperimeter,perimeter,joincount,link,misorientation,neighbours,
paris,shapefactor,toebsd,rotation,axis,Euler,symmetrise,orientation,eq,inverse,BinghamODF,set,
project2FundamentalRegion,annotate,plotAngleDistribution,volumeCompressibility,
YoungsModulus, shearModulus, linearCompressibility, ChristoffelTensor,
velocity,
PoissonRatio,plotx2north,plotx2east,setpref,import,loadOrientation,loadVector3d,brassOrientation,double,EinsteinSum,directionalMagnitude,inv},
sensitive=false, morestring=[b][\emph]', moredelim=[is][\alert]{/+}{+/},
morecomment=[s][\emph]{<options}{>}, morecomment=[l]{\%},
commentstyle=\color{darkgray}\itshape }

\lstset{language=rock}
\lstset{emph={symmetries,concentration},emphstyle={\bf \color{white!25!black}}}


% \definecolor{mydarkgray}{rgb}{0.25,0.3,0.25}
% \lstset{
%   language=Matlab,
%   escapeinside="",
%   keywordstyle=\color{black},
%   numbersep=5pt,
% 	framexleftmargin=5pt,
% 	framexrightmargin=5pt,
% 	framextopmargin=10pt,
% 	framexbottommargin=10pt,
%   moredelim=[is][\color{gray}\itshape]{/*}{*/},
% 	moredelim=[is][\alert]{/+}{+/},
%   fancyvrb=true
% }



 \lstdefinestyle{input}{
 	frame=leftline,
 	framerule=2pt,
 	rulecolor=\color{mygreen},
 	backgroundcolor=\color{lightgray},
 	basicstyle=\smaller\color{black}}

 \lstdefinestyle{output}{
 	frame=leftline,
 	framerule=2pt,
 	rulecolor=\color{gray},
 	backgroundcolor=\color{lightergray},
 	basicstyle=\scriptsize\ttfamily\color{darkgray},
 	emph={ODF,Miller,PoleFigure,Symmetry,Vector3d,rotation,orientation,vector3d,misorientation,tensor},
        emphstyle={\color{blue}\underbar}}


% \definecolor{lightgray}{gray}{0.9} \lstset{backgroundcolor=\color{lightgray}}
% \lstdefinelanguage{rock}{morekeywords=
% {mean,std,hist,save,load,rotate,min,max,get,tensor,loadTensor,
% scale,union,delete,degree,kernel,gethw,getparameter,bandwidth,
% symmetry,Miller,Miller2quat,uniformODF,unimodalODF,fibreODF,xvector,yvector,zvector,
% loadPoleFigure,PoleFigure,plot,plotpdf,plotipdf,calcODF,calcKernel,calcTensor,calcMDF,calcMisorientation,
% vector3d,cross,norm, dot,sum,savefigure,plotDiff,calcerror,
% quaternion,idquaternion,dist,Laue,length,angle,add,subGrid,refine,GridLength,getResolution,getRho,
% polar,S2Grid,santafee,mix2,modalorientation,fourier,textureindex,entropy,volume,
% surface, fibrevolume,uiimport,scatter,cif2symmetry,remove_outlier,
% simulatePoleFigure,SimulateEBSD,grainfun,hold,getFundamentalRegion,symvec,
% FourierODF,find_outlier,bar,copyproperty,plotspatial,hasholes,principalcomponents,
% hullprincipalcomponents,loadEBSD,calcGrains,plotboundaries,plotodf,plotspatial,area,
% aspectratio,grain,grainfun,grainsize,get,plotgrains,plotellipse,plotsubfractions,plotboundary,
% hasholes,hassubfraction,hullarea,hullcentroid,hullperimeter,centroid,borderlength,
% deltaarea,equivalentperimeter,perimeter,joincount,link,misorientation,neighbours,
% paris,shapefactor,toebsd,rotation,axis,Euler,symmetrise,orientation,eq,inverse,BinghamODF,set,
% project2FundamentalRegion,annotate,plotAngleDistribution,volumeCompressibility,
% YoungsModulus, shearModulus, linearCompressibility, ChristoffelTensor, velocity,
% PoissonRatio,plotx2north,plotx2east,setpref}, sensitive=false, morestring=[b][\emph]',
% moredelim=[is][\alert]{/+}{+/}, morecomment=[s][\emph]{<options}{>},
% morecomment=[l]{\%}, commentstyle=\color{darkgray} }

%\lstset{language=rock}
%\lstset{emph={symmetries,concentration},emphstyle={\bf \color{white!25!black}}}

% \lstdefinestyle{input}{%
%         frame=leftline,
% 	framerule=2pt,
% 	rulecolor=\color{darkgray},
% 	backgroundcolor=\color{lightgray},
% 	basicstyle=\footnotesize\color{black},
%         keywordstyle=\bfseries}

% \lstdefinestyle{output}{%
% 	frame=leftline,
% 	framerule=2pt,
% 	rulecolor=\color{gray},
% 	backgroundcolor=\color{lightergray},
% 	basicstyle=\footnotesize\ttfamily\color{black},
% 	emph={ODF,Miller,PoleFigure,Symmetry,tensor,vector3d,orientation},emphstyle={\color{blue}\underbar}}


\newtheorem{Remark}{Remark}
\newtheorem{Proposition}{Proposition}
\renewcommand{\O}{\mathcal O}
\newcommand{\PGroup}{S_\text{Point}}
\newcommand{\Y}{\mathcal Y}
\renewcommand{\angle}{\measuredangle}
\newcommand{\mtex}{{\large \bf{\color{red}M}TEX\,}}%
\newcommand{\MTEX}{{\bf {\color{red}M}TEX\,}}%
\newcommand{\degree}{\ensuremath{^\circ}}   % Gradzeichen
%\newcommand{\degree}{^\circ}
\newcounter{exercisescounter}
\newenvironment{Exercise}{\refstepcounter{exercisescounter}%
  \begin{block}{Exercise \arabic{exercisescounter}}}
  {\end{block}}


\author{R. Hielscher}

\title{{\Huge \bf{\color{red}M}TEX}}
\subtitle{A Texture Calculation Toolbox}
%\titlegraphic{\includegraphics[width=1.5cm]{pic/tu_logo}}

\institute{Faculty of Mathematics,\\
	Chemnitz University of Technology, Germany}

\date{Prag, Mai 31 2011}


\begin{document}

\section{Geometry}



\subsection*{Vector3d}

\begin{frame}[fragile]
  \frametitle{Specimen Directions - The \MTEX Class \texttt{\bf vector3d}}

  Definition:

\begin{lstlisting}
v = vector3d(1,1,1);    % by Cartesian coordinate
v = sph2vec(theta,rho); % by polar coordinates
v = xvector;            % predefined vectors
\end{lstlisting}

  \medskip

  \begin{columns}
    \begin{column}{8.5cm}

      Calculations:

\begin{lstlisting}
v = [xvector,yvector]; w = v(1);
v = 2*xvector-yvector;
\end{lstlisting}

    \medskip

    Basic Functions:

\begin{lstlisting}
cross(v1,v2), dot(v1,v2)
norm(v), sum(v)
vec2sph(v)
\end{lstlisting}
  \end{column}
  \begin{column}{3cm}
   \includegraphics[width=3cm]{pic/vector3d}
  \end{column}
\end{columns}



\end{frame}

\subsection*{quaternion}


\begin{frame}[fragile]
  \frametitle{Rotations - The \MTEX Class \texttt{\bf quaternion}}

Definition:

\begin{lstlisting}
q = quaternion(a,b,c,d);
q = axis2quat(axis,omega);
q = euler2quat(alpha,beta,gamma);
q = euler2quat(phi1,Phi,phi2,'Bunge');
q = Miller2quat([h k l],[u v w],CS);
\end{lstlisting}
%q = idquaternion;

\medskip

\begin{columns}
  \begin{column}{8.5cm}

    Calculations:

\begin{lstlisting}
q = [q1,q2]; q1 = q(1)
q = q1 * q2; w = q * v
\end{lstlisting}

    \medskip

    Basic Functions:

\begin{lstlisting}
omega = rotangle(q)
v = rotaxis(q)
[alpha,beta,gamma]  = quat2euler(q)
\end{lstlisting}

  \end{column}

  \begin{column}{3cm}
    \includegraphics[width=3cm]{pic/quaternion}
  \end{column}

\end{columns}
\end{frame}

\subsection*{Symmetry}
\begin{frame}[fragile]
  \frametitle{Crystal Symmetries - The \MTEX Class \texttt{\bf symmetry}}

  Definition:

\begin{lstlisting}
S = symmetry('triclinic',[a,b,c],[alpha,beta,gamma])
S = symmetry('-3m',[a,b,c],/+'a||x'+/);
S = symmetry('O');
\end{lstlisting}

\medskip

\begin{columns}
  \begin{column}{8.5cm}

Load Symmetry from CIF file:

\begin{lstlisting}
cif2symmetry('quartz.cif')
\end{lstlisting}

\medskip

    Basic Functions:

\begin{lstlisting}
symmetriceVec(SS,v)
dist(CS,SS,q1,q2)
quaternion(CS)
getFundamentalRegion(CS)
\end{lstlisting}
  \end{column}

  \begin{column}{3cm}
    \includegraphics[width=3cm]{pic/sym}
  \end{column}

\end{columns}

\end{frame}

\subsection*{Miller}

\begin{frame}[fragile]
  \frametitle{Crystal Directions - The \MTEX Class \texttt{\bf Miller}}

  Definition:

\begin{lstlisting}
h = Miller(1,0,0,CS);
h = [Miller(1,1,-2,3,CS),Miller(0,1,-1,0,CS)]
h = vec2Miller(v,CS);
\end{lstlisting}

\medskip

\begin{columns}
  \begin{column}{8.5cm}

    Calculations:

\begin{lstlisting}
q * Miller(1,0,0,CS)
q * symeq(Miller(1,0,0,CS))
\end{lstlisting}

    \medskip

    Basic Functions:


    \begin{onlyenv}<1>
\begin{lstlisting}
symeq(h1,h2)
symvec(h)
angle(h1,h2)
plot([h1,h2],'all')
\end{lstlisting}
    \end{onlyenv}

    \begin{onlyenv}<2>
      \lstset{stringstyle=\color{red},emph={antipodal},emphstyle=\em\color{red}}
\begin{lstlisting}
symeq(h1,h2,`antipodal`)
symvec(h,`antipodal`)
angle(h1,h2,`antipodal`)
plot([h1,h2],'all',`antipodal`)
\end{lstlisting}
    \end{onlyenv}

  \end{column}

  \begin{column}{3cm}
    \includegraphics[width=3cm]{pic/miller}
  \end{column}

\end{columns}

\end{frame}

\subsection*{S2Grid}


\begin{frame}[fragile]
  \frametitle{Sets of Specimen Directions - The \MTEX Class \texttt{\bf S2Grid}}

Definition:

\begin{lstlisting}
S2G = S2Grid('regular','resolution',5*degree);
S2G = S2Grid('regular','maxrho',80*degree);
S2G = S2Grid('equispaced','points',10000);
\end{lstlisting}

\medskip

Basic Functions:

\begin{lstlisting}
add, delete, rotate, union, subGrid, refine,
GridLength, getResolution, getRho, getTheta,
polar, vector3d
\end{lstlisting}

\onslide<1->
\begin{center}
  \includegraphics[width=2.5cm]{pic/S2Grid1} \quad
  \includegraphics[width=2.5cm]{pic/S2Grid2} \quad
  \includegraphics[width=2.5cm]{pic/S2Grid3} \quad
  \includegraphics[width=2.5cm]{pic/S2Grid4}
\end{center}

\end{frame}

\subsection*{Exercises}

\begin{frame}

  \begin{Exercise}
    Consider trigonal crystal symmetry.

    \begin{enumerate}[a)]
    \item Find all crystallographic directions symmetrically equivalent to $h
      = (1, 0, \bar 1, 0)$ (Miller indices)!
    \item Find crystallographic directions such that the number of their
      crystallographic equivalent directions on the upper hemisphere (without
      equator) is 1, 3, or 6?
    \item Construct an orientation that rotates the crystallographic
      directions $(0,0,0,1)$ and $(2,\bar 1,\bar 1,0)$ onto the specimen
      directions $(1,0,0)$ and $(0,1,0)$, respectively.
    \item Find all crystallographic equivalent orientations to
      $(45\degree,0\degree,0\degree)$ (Euler angle) and give its Euler angles!
    \item Find all crystallographic equivalent specimen directions to the crystal
      direction $(1,1,\bar 2,1)$ under the orientation
      $(45\degree,0\degree,0\degree)$ (Euler angle)!
    \end{enumerate}

  \end{Exercise}

\end{frame}

%%% Local Variables:
%%% mode: latex
%%% TeX-master: "main"
%%% End:


\end{document}


%%% Local Variables:
%%% mode: latex
%%% TeX-master: .
%%% End:
