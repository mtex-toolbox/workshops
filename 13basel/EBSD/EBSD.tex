% presentation
\documentclass[compress]{beamer}

% handout
%\documentclass[handout]{beamer}

\usepackage{../mtex,fancyvrb,booktabs}
\usepackage{tikz}



\author{R. Hielscher}

\title{Individual Orientation Data}
\subtitle{{\bf{\color{red}M}TEX} - A Texture Calculation Toolbox}

\institute{Faculty of Mathematics,\\
	Chemnitz University of Technology, Germany}

\date{Basel, October 2013}


\begin{document}

\begin{frame}
  \maketitle{}
\end{frame}


\begin{frame}
  \frametitle{Table of Content}

\tableofcontents{}

\end{frame}


\section{Import}


\begin{frame}[fragile]
  \frametitle{Importing EBSD Data}

  \begin{onlyenv}<1-4>
    \begin{lstlisting}[style=input]
ebsd = loadEBSD('example1.ctf')
    \end{lstlisting}
  \end{onlyenv}
  \begin{onlyenv}<2-4>
      \vspace{-0.4cm}
      \begin{lstlisting}[style=input]
plot(ebsd)
    \end{lstlisting}
    \end{onlyenv}
    \begin{onlyenv}<6>
      \vspace{-0.4cm}
      \begin{lstlisting}[style=input]
grainArea = area(grains)
[m,index] = max(grainArea)
    \end{lstlisting}
    \end{onlyenv}
\begin{onlyenv}<1>
\begin{lstlisting}[style=output]
ebsd = /+EBSD+/ (show methods, plot)
  file name: example1.txt
  Properties: bands, bc, bs, error, mad, x, y
  Phase Orientations    Mineral Color Symmetry  reference frame
      0        58485 notIndexed
      1       152345 Forsterite  blue      mmm
      2        26058  Enstatite green      mmm
      3         9064   Diopside   red      2/m  X||a*, Y||b*, Z||c
  \end{lstlisting}
    \end{onlyenv}
\begin{onlyenv}<3>
\begin{lstlisting}[style=output]
grains = /+Grain2d-Set+/ (show methods, plot)
  file name: mylonite.txt
  EBSD properties: x, y, mis2mean
  Phase Grains Orient.    Mineral Symmetry Cryst. reference frame
      1   1951    3444   Andesina       -1          X||a*, Z||c
      2    776    3893     Quartz      -3m   X||a*, Y||b,  Z||c*
      3    305     368    Biotite      2/m   X||a*, Y||b*, Z||c
      4   1641    4781 Orthoclase      2/m   X||a*, Y||b*, Z||c
  \end{lstlisting}
\end{onlyenv}
\begin{onlyenv}<6>
\begin{lstlisting}[style=output]
m =
   1.4985e+06
index =
        1795
\end{lstlisting}
\end{onlyenv}
\end{frame}




\section{Data Correction}



\section{ODF Estimation}



\section{Grain Reconstruction}


\section{Misorientations}






\subsection*{Visualization}

\begin{frame}[fragile]
  \frametitle{Visualize EBSD Data in \MTEX}

  \begin{columns}
    \begin{column}{8.5cm}

Spatial plots of EBSD data
\begin{onlyenv}<1 |handout:0>
\begin{lstlisting}
/+plot(ebsd)+/
plot(ebsd,'colorcoding','hkl')
plot(ebsd,'property','phase')
plot(ebsd,'property','mad')
\end{lstlisting}
\end{onlyenv}
%
\begin{onlyenv}<2 |handout:0>
\begin{lstlisting}
plot(ebsd)
/+plot(ebsd,'colorcoding','hkl')+/
plot(ebsd,'property','phase')
plot(ebsd,'property','mad')
\end{lstlisting}
\end{onlyenv}
%
\begin{onlyenv}<3 |handout:0>
\begin{lstlisting}
plot(ebsd)
plot(ebsd,'colorcoding','hkl')
/+plot(ebsd,'property','phase')+/
plot(ebsd,'property','mad')
\end{lstlisting}
\end{onlyenv}
%
\begin{onlyenv}<4 |handout:0>
\begin{lstlisting}
plot(ebsd)
plot(ebsd,'colorcoding','ipdf')
plot(ebsd,'property','phase')
/+plot(ebsd,'property','mad')+/
\end{lstlisting}
\end{onlyenv}
%
\begin{onlyenv}<5-|handout:1>
\begin{lstlisting}
plot(ebsd)
plot(ebsd,'colorcoding','ipdf')
plot(ebsd,'property','phase')
plot(ebsd,'property','mad')
\end{lstlisting}
\end{onlyenv}

Color-Codings: ipdf, hkl, bunge, angle

\medskip

\pause
\pause
\pause
\pause

Scatter plots in Rodrigues space or axis angle space
\begin{lstlisting}
scatter(ebsd)
\end{lstlisting}

\medskip

\pause

Scatter plots in pole figures, inverse pole figures, or ODF sections
\begin{lstlisting}
		/+plotpdf(ebsd,[Miller(0,0,1),...])+/
		plotipdf(ebsd,vector3d(1,0,0))
		plotodf(ebsd)
\end{lstlisting}

    \end{column}

    \begin{column}{3.5cm}
      \only<5 |handout:0>{%
      \includegraphics[width=3.5cm]{pic/ebsdscatter}%
      }%
      \only<6|handout:0>{%
      \includegraphics[width=3.2cm]{pic/EBSDpdf}%
      }%
      \only<1|handout:1>{%
      \includegraphics[height=7.5cm]{pic/ebsdsmall}%
      }%
      \only<2|handout:1>{%
      \includegraphics[height=7.5cm]{pic/rawhkl}%
      }%
      \only<3|handout:0>{%
      \includegraphics[height=7.5cm]{pic/ebsdphase}%
      }%
      \only<4|handout:0>{%
      \includegraphics[height=7.5cm]{pic/ebsdmad}%
      }%
    \end{column}
  \end{columns}
\end{frame}



\subsection*{EBSD Data Correction}


\begin{frame}[fragile]
 \frametitle{EBSD Data Correction}

  \begin{columns}
   \begin{column}{8cm}
	Rotate, shift and flip EBSD data
\begin{lstlisting}
ebsd = rotate(ebsd,5*degree)
ebsd = fliplr(ebsd)
\end{lstlisting}

        \pause
	\medskip

        Remove bad values
\begin{lstlisting}
mad = get(ebsd,'mad')
ebsd = delete(ebsd,mad > 0.75)
\end{lstlisting}

        \pause
	\medskip

       Omit small grains
\begin{lstlisting}
[grains,ebsd] = calcGrains(ebsd)
gs = grainsize(grains)
ebsd = link(ebsd,grains(gs > 10))
\end{lstlisting}

   \end{column}

    \begin{column}{4cm}
      \only<1|handout:1>{%
      \includegraphics[height=7.5cm]{pic/rawrot}%
      }%
      \only<2|handout:0>{%
      \includegraphics[height=7.5cm]{pic/rawmad}%
      }%
      \only<3|handout:0>{%
      \includegraphics[height=7.5cm]{pic/rawSmallGrains}%
      }%

    \end{column}

 \end{columns}

\end{frame}

\subsection*{Grains Analysis}

\begin{frame}[fragile]
  \frametitle{Grain Analysis}

  \begin{columns}
    \begin{column}{8.5cm}

      \medskip

      Grain Detection

\begin{lstlisting}
[grains ebsd] = calcGrains(ebsd,...
    'angle',10*degree)
\end{lstlisting}

% \begin{tabular}{ll}
%   Option & Value \\
%   \toprule
%   angle &  threshold angle\\
%   distance &  max distance\\
%   augmentation & bounding box\\
%   unitcell &  use a unitcell
% \end{tabular}

\medskip

\pause

Plot grain boundaries:
\begin{lstlisting}
hold on
plotboundary(grains,/+<options>+/)
hold off
\end{lstlisting}

\begin{tabular}{ll}
  Option & Value \\
  \toprule
  color & red, blue, black, green\\
  linewidth & number of points\\
  property & phase, missorientation\\
\end{tabular}

\end{column}

\begin{column}{3.5cm}
  \includegraphics[height=7.5cm]{pic/ebsdgrains}
\end{column}
\end{columns}

\end{frame}


\subsection*{Grain Operations}


\begin{frame}[fragile]
  \frametitle{Grain Operations}

  \begin{columns}
    \begin{column}{8.5cm}

      Plotting grains
      \begin{onlyenv}<1 |handout:1>
\begin{lstlisting}
plot(grains,'property','orientation')
\end{lstlisting}
      \end{onlyenv}

      \begin{onlyenv}<2- |handout:0>
\begin{lstlisting}
plot(grains,'property',/+'phase'+/)
\end{lstlisting}
      \end{onlyenv}

      \pause
      \pause

\bigskip

      Access grain properties
\begin{lstlisting}
phase  = get(grains,'phase')
grain1 = grains(phase == 1)
ori    = get(grain1,'orientation')
\end{lstlisting}

\pause

\bigskip

Access individual grains
\begin{lstlisting}
plot(grain(200))
\end{lstlisting}

\pause
\bigskip

smooth grain boundaries
\begin{lstlisting}
grains = smooth(grains,2,'S')
\end{lstlisting}



% Interconnection with EBSD Data
% \begin{lstlisting}
% gS = grainSize(grains)
% largeGrains = grains(gs>10)
% large_ebsd = link(largeGrains,ebsd)
% \end{lstlisting}

    \end{column}
    \begin{column}{3.5cm}
      \only<1|handout:1>{%
      \includegraphics[height=7.5cm]{pic/ebsdgrainorientation}%
      }%
      \only<2-3|handout:0>{%
      \includegraphics[height=7.5cm]{pic/ebsdgrainsphase}%
      }%
      \only<4|handout:0>{%
      \includegraphics[height=8cm]{pic/grain.png}%
      }%
      \only<5|handout:0>{%
      \includegraphics[height=8cm]{pic/grainSmoothed.png}%
      }%
      \only<6|handout:0>{%
      \includegraphics[height=8cm]{pic/smoothedGrains.png}%
      }%
    \end{column}
  \end{columns}

\end{frame}


%

\subsection*{Grain Properties}



\begin{frame}[fragile]
  \frametitle{Geometrical Grain Properties}

Basic functions on grain geometry
%[basicstyle=\footnotesize]
\begin{lstlisting}
grainsize, area, perimeter, volume, surface,
shapefactor,paris
\end{lstlisting}

\begin{columns}[t]
  \begin{column}[T]{6.5cm}

\medskip

  grainsize distribution
\begin{lstlisting}
A = area(grains);
bar( hist(A,exp(-1.5:6.5)) )
\end{lstlisting}

\medskip

other properties:
\begin{lstlisting}[basicstyle=\footnotesize]
hasholes,hassubfraction
\end{lstlisting}

	\end{column}
	\begin{column}[T]{5cm}
		\includegraphics[width=5cm]{pic/grh}
	\end{column}
\end{columns}

select grains by properties
\begin{lstlisting}
grains = grains( area(grains) > mean(area(grains)) )
\end{lstlisting}

\end{frame}


%



\subsection*{EBSD - to - ODF Reconstruction}


\begin{frame}[fragile]
  \frametitle{EBSD to ODF Reconstruction in \MTEX}

  \mtex uses kernel density estimation to compute an ODF from EBSD data. The
  sensitive parameter of this method is the kernel function.

\medskip

  Syntax:
  \begin{alertenv}
\begin{lstlisting}
psi = calcKernel(grains,'phase',1)
odf = calcODF(ebsd,'phase',1,'kernel',psi)
\end{lstlisting}
  \end{alertenv}

\medskip

\begin{tabular}{ll}
  Option & Description \\
  \toprule
  phase & phase to be considered\\
  kernel & kernel to be used\\
  halfwidth & halfwidth for the default de la Vall\'ee Poussin kernel\\
  Fourier & Fourier representation of the ODF\\
  resolution &  resolution of the binning\\
  exact & no binning
\end{tabular}


\end{frame}

\subsection*{Misorientation Analysis - TODO}

\begin{frame}[fragile,fragile]
  \frametitle{Misorientation Analysis - TODO}

  compute discrete misorientations

\begin{lstlisting}
misori = calcMisorientation(grains)
\end{lstlisting}

  \medskip
  \pause

plot misorientation angle distribution
\begin{lstlisting}
plotAngleDistribution(misori)
\end{lstlisting}

  \medskip
  \pause

compute misorientation distribution function
\begin{lstlisting}
mdf = calcMDF(grains)
\end{lstlisting}

  \medskip
  \pause


compute uncorrelated misorientation distribution function
\begin{lstlisting}
odf = calcODF(ebsd)
mdd = calcMDF(odf)
plotAngleDistribution(mdf)
\end{lstlisting}

\end{frame}


\subsection*{EBSD3d}

\begin{frame}[fragile]
  \frametitle{3D EBSD Data}


\begin{lstlisting}
plot(ebsd)
\end{lstlisting}

\centering{
\only<1|handout:1>{\includegraphics[height=7cm]{pic/3d_slice}}
}

\end{frame}


\begin{frame}[fragile]
\frametitle{3D EBSD Data}

\begin{lstlisting}
plot(grains)
\end{lstlisting}

\centering{
\only<1|handout:1>{\includegraphics[height=7cm]{pic/3dgrains}}
\only<2|handout:0>{
\includemovie[poster,toolbar,label=grains_smooth2,text=(5 largest grains),3Dviews2=pic/views.txt,3Djscript=pic/grains_explode.js]{11cm}{8cm}{pic/grains_smooth2.u3d}
}
}

%\only<3|handout:0>{
%\includemovie[poster,toolbar,label=grains_vox,text=(5 largest grains),3Dviews2=pic/views.txt,3Djscript=pic/grains_explode.js]{11cm}{8cm}{pic/grains_vox.u3d}
%}
%\only<4|handout:0>{
%\includemovie[poster,toolbar,label=grains_smooth,text=(5 largest grains),3Dviews2=pic/views.txt,3Djscript=pic/grains_explode.js]{11cm}
%{8cm}{pic/grains_smooth.u3d}
%}

\end{frame}


\subsection*{Exercises}

\begin{frame}

  \begin{Exercise}
    \begin{enumerate}[a)]
      \item Load the EBSD data:
      \texttt{data/ebsd/85\_829grad\_07\_09\_06.txt}.
      \item Perform grain detection with a different thresholds.
      \item Plot the EBSD data together with the grain boundaries.
      \item Visualize the grains with their mean orientations.
      \item Compute and visualize the grain size distribution.
      \item Explore the geometric properties of the grains. Is there any
      relationship between the size and the mad of the grains?
    \end{enumerate}
  \end{Exercise}

  \begin{Exercise}
    \begin{enumerate}[a)]
      \item Estimate an ODF from the above EBSD data.
      \item Visualize the ODF and some of its pole figures.
      \item Explore the influence of the halfwidth on the kernel
      density estimation by looking at the pole figures! Which kernel is the best?
      \item Compute the optimal kernel by the command \texttt{calcKernel}.
    \end{enumerate}
  \end{Exercise}


\end{frame}

%\begin{frame}

  % \begin{block}{Exercises 6}
  %   \begin{enumerate}
  %   \item Start with an arbitrary model ODF!
  %   \item Compute the volume portion of the ODF within a range of $20\degree$
  %     of the modalorientation and compare it to the corresponding volume of
  %     the uniform ODF!
  %   \item Simulate EBSD data from this ODF with 10.000 orientations.
  %   \item Plot pole figures from the EBSD data and compare them with the pole
  %     figures from the model ODF.
  %   \item Compute the volume portion of the estimated ODF within a range of
  %     $20\degree$ of the modalorientation and compare it to model ODF!
  %   \item Perform these investigations for different sample sizes!
  %   \end{enumerate}
  % \end{block}


%   \begin{Exercise}
%     \begin{enumerate}[a)]
%     \item Estimate an overall grain-ODF and compare it with the ODF of
%       original EBSD data
%     \item Visualize the textureindex of each grain
%     \item Compare the intrinsic misorientation of individual grains to its
%       texture properties, how does the threshold angle affect this?
%     \item Investigate the misorientation of neighbours, how does the threshold
%       angle influence it?
%     \end{enumerate}
%   \end{Exercise}

%\end{frame}



\end{document}


% \begin{frame}[fragile]
% \begin{lstlisting}[mathescape=true]
% rot = rotation('Euler',$\phi_1$,$\Phi$,$\phi_2$,'ZXZ')
% rot = rotation('Euler',$\alpha$,$\beta$,$\gamma$,'ABG')
% rot = rotation('axis',v,'angle',omega)
% rot = rotation('map',$u_1$, $v_1$, $u_2$, $v_2$) % $\mathbf{gu}_1=\mathbf{v}_1,\mathbf{gu}_2=\mathbf{v}_2$
% \end{lstlisting}

% \begin{columns}
%   \begin{column}{8.5cm}

%     Calculations:

% \begin{lstlisting}
% v = rot * u    % apply rot to u
% rot = rot1 * rot2
% \end{lstlisting}

%     Basic Functions:

% \begin{lstlisting}[mathescape=true]
% angle(rot), axis(rot)
% angle(rot1,rot2), inverse(rot)
% [$\phi_1$,$\Phi$,$\phi_2$] = Euler(rot)
% [$\alpha$,$\beta$,$\gamma$] = Euler(rot,'ABG')
% \end{lstlisting}

%   \end{column}

%   \begin{column}{3cm}
%     \includegraphics[width=3cm]{pic/quaternion}
%   \end{column}

% \end{columns}
% \end{frame}


% \subsection*{Symmetry}
% \begin{frame}[fragile]
%   \frametitle{Crystal Symmetries - The \MTEX Class \texttt{\bf symmetry}}

%   Definition:

% \begin{lstlisting}
% S = symmetry('triclinic',[a,b,c],[alpha,beta,gamma])
% S = symmetry('-3m',[a,b,c],/+'X||a*'+/);
% S = symmetry('-3m',[a,b,c],'X||a','Z||c*');
% S = symmetry('O');
% \end{lstlisting}

% \medskip

% \begin{columns}
%   \begin{column}{8.5cm}

% Load Symmetry from CIF file:

% \begin{lstlisting}
% symmetry('quartz.cif')
% \end{lstlisting}

% \medskip

%     Basic Functions:

% \begin{lstlisting}
% symmetrise(v,S)
% symmetrise(rot,CS,SS)
% rotation(S)
% project2FundamentalRegion(v,CS)
% project2FundamentalRegion(rot,CS,SS)
% \end{lstlisting}
%   \end{column}

%   \begin{column}{3cm}
%     \includegraphics[width=3cm]{pic/sym}
%   \end{column}

% \end{columns}

% \end{frame}

% \subsection*{Miller}

% \begin{frame}[fragile]
%   \frametitle{Crystal Directions - The \MTEX Class \texttt{\bf Miller}}

%   Definition:

% \begin{lstlisting}
% m = Miller(u,v,w,CS,'uvw');
% m = Miller(h,k,i,l,CS,'hkl');
% m = [Miller(1,1,-2,3,CS),Miller(0,1,-1,0,CS)]
% \end{lstlisting}

% \medskip

% \begin{columns}
%   \begin{column}{8.5cm}

%     Calculations:

% \begin{lstlisting}
% h1 + h2
% rot * h     % apply rot on h
% \end{lstlisting}

%     \medskip

%     Basic Functions:

%     \begin{onlyenv}<1>
% \begin{lstlisting}
% eq(h1,h2), angle(h1,h2)
% symmetrise(h)  %get all equivalent
% plot([h1,h2],'all')
% \end{lstlisting}
%     \end{onlyenv}

% 	\end{column}

%   \begin{column}{3cm}
%     \includegraphics[width=3cm]{pic/miller}
%   \end{column}
% \end{columns}


% 	\begin{onlyenv}<2>

% 		\lstset{stringstyle=\color{red},emph={antipodal},emphstyle=\em\color{red}}

% \begin{lstlisting}
% get(h,'hkl')
% eq(h1,h2,`antipodal`), angle(h1,h2,`antipodal`)
% symmetrise(h,`antipodal`)
% plot([h1,h2],'all',`antipodal`)
% \end{lstlisting}


%     \end{onlyenv}


% \end{frame}



% \begin{frame}[fragile]
%   \frametitle{Orientations - The \MTEX Class \texttt{\bf orientation}}

% Definition:

% \begin{lstlisting}[mathescape=true]
% ori = orientation(rot,CS,SS)
% ori = orientation('Euler',$\phi_1$,$\Phi$,$\phi_2$,CS,SS)
% ori = orientation('Euler',$\alpha$,$\beta$,$\gamma$,'ABG',CS,SS)
% ori = orientation('Miller',[h k l],[u v w],CS,SS)
% ori = orientation('brass',CS,SS)
% \end{lstlisting}

% \begin{columns}
%   \begin{column}{8.5cm}

%     Calculations:

% \begin{lstlisting}
% r = ori * h, h = ori \ r
% ori = rot * ori
% \end{lstlisting}

%     Basic Functions:

% \begin{lstlisting}[mathescape=true]
% eq(ori1,ori2), angle(ori1,ori2)
% symmetrise(ori), angle(ori)
% project2FundamentalRegion(ori,ref_ori)
% [$\phi_1$,$\Phi$,$\phi_2$]  = Euler(ori)
% \end{lstlisting}

%   \end{column}

%   \begin{column}{3cm}
%     \includegraphics[width=3cm]{pic/quaternion}
%   \end{column}

% \end{columns}
% \end{frame}

2732626
